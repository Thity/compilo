\section{Semicolon Inference}

\textnormal{The semicolon is not required anymore at the end of an instruction
  when the next instruction is in another line, like in scala. It is still needed
  for a statement which precedes another instruction on the same line.}

\subsection{Example}

\begin{lstlisting}
class Matrix2 {
  def meaninglessTransformation1() : Matrix2 = {
    var bias : Matrix2; bias.init(1,0,0,1);
    return this.times(bias1);
  }
}
\end{lstlisting}
\textmd{Becomes}
\begin{lstlisting}
class Matrix2 {
  def meaninglessTransformation1() : Matrix2 = {
    var bias : Matrix2; bias.init(1,0,0,1)
    return this.times(bias1)
  }
}
\end{lstlisting}

\subsection{Implementation}

%% We first thought we had to treat this token in the grammar, but it was a very
%%  bad idea not far from impossible.
%% We first thought we had to treat this token in the grammar, but it was a bad
%% idea, and all our approaches to make the grammar in LL1 failed.

\textnormal{Up until now, line jumps were ignored by the lexer as any
  whitespace. However, in order to add semicolon inference, we consider them as
  semicolon when needed. We have defined a new token for line jumps in the lexer.
  We first thought we had to treat this token in the grammar, but it was a very
  bad idea not far from impossible. The semicolon inference has to be done after
  the lexing and before the parsing. We then created a huge method in Parser.scala
  which transforms the list of Tokens before parsing it. We helped ourselves
  with the following simple list of rules for scala semicolon inference \cite{rules_semicolon} :} 

\begin{itshape}
  A line ending is treated as a semicolon unless one of the following conditions
  is true :
  \begin{itemize}
  \item The line in question ends in a word that would not be legal as the end
    of a statement, such as a period or an infix operator.
  \item The next line begins with a word that cannot start a statement.
  \item The line ends while inside parentheses (\ldots) or brackets [\ldots], because
    these cannot contain multiple statements anyway.
  \end{itemize}
\end{itshape}

\textnormal{We noticed that two cases were not correctly handled by these rules
  :}
\begin{enumerate}
\item The multiple-lines if-statements, where we don't want the line jump to be
  treated like a semicolon :
  \begin{lstlisting}
    if(condition) // no semicolon
      do(expression) // semicolon
  \end{lstlisting}
\item The closing braces on the same line than the statement, where we want a
  semicolon before '\}' even if there is no line jump :
  \begin{lstlisting}
    if(condition) { println("true"); }
    else { println("false"); }
  \end{lstlisting}
\end{enumerate}

\textnormal{The 'semicolonInference' methods takes the list of tokens resulting
  from the token iterator then returns an adapted list with no more LINEJUMP 
  token. This transformation consists of the following steps :}


\begin{lstlisting}
  def semicolonInference(list: List[Token]): List[Token]
\end{lstlisting}

\begin{itemize}
\item Add the line jump token before each closing brace token to solve the 2nd
  issue described above.
\item Remove line jumps before non-starting tokens and after non-ending tokens
\item Remove line jumps contained in parenthesis or brackets
\item Remove line jumps placed after an if-condition (1st issue above)
\item Transform each remaining line jump into semicolon
\end{itemize}

\textnormal{This implementation is backwards compatible, as we can mix the old
  syntax (semicolon) and the new syntax (no semicolon) at the same time.}

