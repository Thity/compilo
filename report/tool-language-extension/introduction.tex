\section{Introduction}

\textnormal{During the semester, we implemented an interpreter and a compiler in
  scala for a simple object oriented programming language : Tool. The coding
  constraints for this language are far more restrictive as in scala. We chose
  to add some improvements, generally inspired from scala, to this language and
  permit more coding freedom.}
\textnormal{In order to add more coding freedom, we implemented 3 different problems:}

\begin{itemize}
\item Semicolon interference
\item Infix Operations
\item Parameterless calls
\end{itemize}

\textnormal{To write the initial compiler and interpreter we divided the problem
  in a 5 stage pipeline.}

\begin{lstlisting}
Lexer $\rightarrow$ Parser $\rightarrow$ Name Analysis $\rightarrow$ Type Check
$\rightarrow$ Code Constructor
\end{lstlisting}

\subsection{Lexer}
\textnormal{The Lexer is the step where we filter the useful parts of the input
  text. For each chain of characters we will define a Token, for example :}
\vfill
\begin{lstlisting}
; -> SEMICOLON
while -> WHILE
charly -> ID
\end{lstlisting}
\vfill
\textnormal{The challenges of the lexer are :}
\begin{itemize}
\item Define the tokens we need
\item Filter white spaces and comments
\end{itemize}

\textnormal{After the lexer runs, we get a chain of tokens as the result. For
  example :}
\begin{lstlisting}
def foo(): Int ...
\end{lstlisting}
\textmd{Becomes}
\begin{lstlisting}
DEF() ~ ID() ~ LBRACKET() 
~ RBRACKET() ~ COLON() ~ INT()...
\end{lstlisting}

\textnormal{In the case of our extension, we had to add the following new token
  for the semicolon interference: LINEJUMP that will make a token out of
  '\textbackslash r' and '\textbackslash n' characters.}

\subsection{Parser}
\textnormal{Parser is the step where we define the grammar of the language and
  then transform it into an abstract form. The grammar is the set of rules that
  the code structure must follow. The best way to parse a list of tokens is to
  generate an Abstract Syntax Tree (AST).}

\textnormal{The challenges of the parser are :}
\begin{itemize}
\item In order to parse the tree in linear time, we must have the grammar in
  LL1, this might be sometimes impossible unless we put some conditions in the
  grammar
\item Create the tree nodes depending on the those conditions.
\end{itemize}

\textnormal{After the parser runs, we get an Abstract Syntax Tree (AST).\\}

\textnormal{In the case of our extension, many changes had to be done in the
  grammar and constructor that we will explain in the other chapters.}

\subsection{Name Analysis}
\textnormal{Name Analysis is the step where we check if each variable call has
  the right conditions.}

\textnormal{In order to do this we pass throught the tree and create a 'Symbol'
  for each identifier. When creating these symbols we check that conditions as
  following hold:}

\begin{itemize}
\item Scope of the variable, that means for example if we define it in a
  function, it must only be accesed from that function. 
\item Existence, which means that an accessed variable must be defined before.
\item Etc..
\end{itemize}

\textnormal{In the case of our extension, we didn't make any changes for this
  part. }

\subsection{Type Check}

\textnormal{Type Checking is the step where we check if the types of expressions
  match what we expect them to be. For example: }

\begin{lstlisting}
  program Program {
    var b : B;
    var array : Int[];
    do(b.foo() + 1); // success, check return of super class
    do(array + b.foo()); // fail, sum of int with an int array
  }
  class A { def foo(): Int = {<@$\ldots$@>}}
  class B extends A {<@$\ldots$@>}
\end{lstlisting}
\\
\textnormal{The challenge of the type checker is to go through the program from
  the Leafs to the Root and to check that the Nodes have the right types.\\}
\\
\textnormal{In the case of our project, we changed it to allow +, -, *, /
  operations for Class Types.\\}
\\
\textnormal{The Parsing, Name analyzis and Type Checking steps check if a
  program is valid or not. If we have a valid program we are ready for the last
  step : code generation.\\}

\vfill
\subsection{Code Generation}
\textnormal{In this last step we end up generating JVM code from the final
  structure we created, to be executable by the JVM. We used the 'Cafebabe'
  library, which makes the JVM instructions creation much more easier. For
  example:}

\begin{lstlisting}
  Node(Plus(Variable(id1),Variable(id2))
\end{lstlisting}
\textnormal{Generates}
\begin{lstlisting}
  iload_n1
  iload_n2
  iadd
\end{lstlisting}

\textnormal{In the case of our project, we changed it for the infix operators
  part.}


