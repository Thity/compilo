\section{Parameterless Calls}
\textnormal{To define or call a method which takes no parameter, you don't need
  to put parenthesis anymore. This works for 'new' instances too.}

\subsection{Example}

\begin{lstlisting}
  new Matrix2();
  ...
  class Matrix2 {
    def isInversible() : Boolean = {
      var determinant : Int;
      determinant = this.getDeterminant();
      return determinant != 0;
    }
  }
\end{lstlisting}
\textmd{Becomes}
\begin{lstlisting}
  new Matrix2;
  ...
  class Matrix2 {
    def isInversible : Boolean = {
      var determinant : Int;
      determinant = this.getDeterminant;
      return determinant != 0;
    }
  }
\end{lstlisting}

\subsection{Implementation}

\textnormal{This feature was totally implemented on the parser. We had to change
  the definition of method declarations, dot operation and new expression on the
  grammar, to make another option for the arguments as following. It gives the
  possibility to not put parenthesis for the cases we don't need to.}

\lstset{escapeinside={<@}{@>}}
\begin{lstlisting}
<@$\ldots$@>
'MethodDeclaration ::= DEF() ~ 'Identifier ~ <@\colorbox{green}{'ParamsOpt}@>
~ COLON() ~ 'Type ~ EQSIGN() ~ LBRACE()  
~ 'VarDecs ~ 'Stmts ~ RETURN() ~ 'Expression
~ SEMICOLON() ~ RBRACE(),
'ParamsOpt ::= LPAREN() ~ 'Params ~ RPAREN()
| epsilon(),
<@$\ldots$@>
'ExprTerm ::= <@$\ldots$@> | NEW() ~ 'NewEnd | <@$\ldots$@> 
'NewEnd ::= <@\colorbox{green}{'Identifier ~ 'ParenOpt}@>
  | INT() ~ LBRACKET() ~ 'Expression ~ RBRACKET(),
'ParenOpt ::= LPAREN() ~ RPAREN()
  | <@\colorbox{green}{epsilon()}@>,
<@$\ldots$@>
\end{lstlisting}

\textnormal{Similar changes were done in the non-LL1 grammar and the
  constructors have been adjusted.}